\documentclass[
11pt,%
tightenlines,%
twoside,%
onecolumn,%
nofloats,%
nobibnotes,%
nofootinbib,%
superscriptaddress,%
noshowpacs,%
centertags]%
{revtex4}
\usepackage{ljm}



%




 % for running heads
% % for running heads %for running heads
%

\setcounter{page}{1}

\begin{document}
\titlerunning{Power filtration on morphisms of formal group law}
\authorrunning{I.\,I.~Nekrasov}

\title{Power Filtration On Morphisms of Formal Group Law}

\author{\firstname{I. I.}~\surname{Nekrasov}}
\email[E-mail: ]{geometr.nekrasov@yandex.ru}
\affiliation{St Petersburg State University, Chebyshev Laboratory, St Petersburg, 14th line of V.O., 29b}


\firstcollaboration{(Submitted by  A.M. Elizarov) }

\received{October 6, 2017}

\begin{abstract}
The height filtration on the stack of formal groups $\mathcal{M}_{FG}$ is well known. We explore analogous filtration on a set of morphisms of formal group laws, which extends to the stack $\mathcal{M}_{FG}$. It is correctly defined colimit object for this filtration which can be identified with the colimit $\mathcal{M}_{FG,\infty}$.  As a corollary we prove explicitly density of additive formal group in any group law.
\end{abstract}
\subclass{11S31, 14L05}
\keywords{Formal group laws, height filtration}

\maketitle



\section{Introduction}
\label{intro}

Formal group laws is a powerful tool for local problems in number theory. It is classically known that kernels of isogenies are the main reason for all arithmetic fenomenas concerning formal groups. For instance, $p$-divisible groups were invented exactly as a generalization of the Tate module (kernel of $\infty$-isogeny) of formal groups, not formal groups itself. Also the definition of the stack $\mathcal{M}_{FG}$ \cite{Smith} incarnates the idea of high importance only of the isogenies among all homomorphisms of formal group laws.

It was shown that additive group isomorphic modulo a Tate module to a subgroup of any formal group by highly abstract reasonings \cite{Smith}. We give absolutely direct method of constructing such groups.

Section \ref{prelim} contains preliminaries for further sections. In section \ref{corre} we following J.Lubin develop lattice--finite subgroups correspondence theory on the level of full functors. In section \ref{filt} we develop power filtration on morphisms of formal group laws, which in case of endomorphism ring give us an explicit subgroup isomorphic to the additive group law.

\section{Preliminaries}
\label{prelim}
We fix a prime number $p$ and a local field $K$, which we assume to be a finite extension of the field of p-adic numbers $\mathbb{Q}_{p}$. Ring of integers and maximal ideal of the field $K$ we denote by $\mathcal{O}_{K}$ and $\mathfrak{m}_{K}$ respectively. It is known that we can continue discrete valuation $v_{\mathbb{Q}_{p}}$ from $\mathbb{Q}_{p}$ to $K$ in the unique way. Then by a uniformizer $\pi_{K}$ of maximal ideal $\mathfrak{m}_{K}$ we mean any element with valuation equals 1.

By $FGL$ we denote the category of all one dimensional formal group laws \cite{Tate}. Its quotient (in stack terminology) by a group of all isomorphisms we get $\mathcal{M}_{FG}$. We assume that all formal group laws are defined over the ring $\mathcal{O}_{K}$. For any such group law $F$ and maximal ideal $\mathfrak{m}_{L}$ of any extension $L/K$ we form an actual group --- group of points $F(\mathfrak{m}_{L})$ with addition given by $a +_{F} b := F(a,b)$.

We refer to \cite{Lubin1} for structural results on ring of endomorphisms of formal group law.  For any extension $L/K$ abelian group $F(\mathfrak{m}_{L})$ has a natural $End_{\mathcal{O}_{K}}(F)[Gal(L/K)]$--module structure.

Also we remind that the notion of height \cite{Manin} is well-defined for formal group laws. Moreover, the height induces a natural filtration not only on the category $FGL$, but on a stack $\mathcal{M}_{FG}$
$$FGL_{\geq 1} \supset FGL_{\geq 2} \supset \dots \supset FGL_{\infty},$$
$$\mathcal{M}_{FG, \geq 1} \supset \mathcal{M}_{FG, \geq 2} \supset \dots \supset \mathcal{M}_{FG, \infty}.$$
By $FGL_{fin}$ we denote a full subcategory of formal group laws of \textbf{finite} height. It is well known that all groups of infinite height are isomorphic to the additive group law $F_{a}(X,Y)$.

Set of homomorphisms between any two formal groups $Hom_{FGL}(F,G)$ is not empty if and only if the corresponding formal groups $F$ and $G$ have the same height.

\section{Lattice--subgroups correspondence for formal group laws}
\label{corre}
The most complete exposition on the results in classical lattice--subgroups correspondence can be found in the paper \cite{Lubin} of J.Lubin. We notice that results of this section is just an incarnation of the general idea of correspondence between sets and corresponding annihilators in a context of formal group laws.

The only endomorphisms of any formal group law $F(X,Y)$ over $\mathcal{O}_{K}$ without inverse are of the form $[\pi^{t}]_{F}(X)$ for $t\in \mathbb{N}$, where $\pi$ is a prime element of an endomorphisms ring\footnote{Here and below $End(F)$ is an absolute endomorphisms ring \cite{Lubin1} for formal group law $F(X,Y)$.} $End_{\mathcal{O}_{K}}(F) = End(F) \cap \mathcal{O}_{K}$. Kernels of this endomorphisms $W_{F}^{m}(K) = \left\{ x \in \mathfrak{m}_{K}:\;[\pi^{m}]_{F}(x) = 0\right\}$ actually are $End_{\mathcal{O}_{K}}(F)$ and $Gal(L/K)$--modules. Union of all this kernels we denote by $W_{F}^{\infty}(K)$.

Let $V(F)$ be a space of all sequences $(a_{0}, a_{1}, \dots)$ such that $[\pi^{k}]_{F}(a_{0})~=~0$ for some $k\geq 1$ and $ a_{i}=[p]_{F}(a_{i+1})$. The Tate module $T(F)$ of formal group law is a subspace of $V(F)$ with $a_{0}$ equals $0$.

First form  of the following theorem which describes a correspondence between finite subgroups in $F$ and sublattices of $V(F)$ appeared in \cite{Lubin}. We present a natural generalization of the theorem to a level of equivalence of the corresponding functors. The striking difference of the statement from the original one is that sets of morphisms are expanded to a maximal one.

\begin{theorem}\label{cor}
We define two functors $\mathcal{L}_{\cdot}$ and $\mathcal{F}_{\cdot}$ from the category $FGL_{fin}$, which on formal group $F$ equals to
\begin{itemize}
\item a category $\mathcal{L}_{F}$ as a category of sublattice $L$ of $V(F)$, which is contained and contains in $p$-homotheties of $T(F)$ with morphisms induced by $W_{F}^{\infty}(K)$--linear maps of $V(F)$;
\item a category $\mathcal{F}_{F}$ as a category of all finite subgroups of a formal group law $F$ with morphisms induced by endomorphisms of formal group $F$.
\end{itemize}
Then functors $\mathcal{L}_{\cdot}$ and $\mathcal{F}_{\cdot}$ are naturally equivalent.

Natural equivalence is given on objects by taking any lattice $L$ to values set of the first non-zero coordinates of all points from $L$.
\end{theorem}

\begin{proof}
For any morphisms between lattices we just set corresponding morphisms on values groups of first non-zero coordinate.

But for homomorphisms of finite subgroups we should use the expanded form of the Theorem 4 from \cite{Tate}, i.e. any such homomorphism we can present as an "analytic" homomorphism (coming from the formal group structure on $F(X,Y)$) and then pull it to a morphism of lattices.
\end{proof}

\section{Height filtration on homomorphisms}
\label{filt}
In this section we consider all formal group law over algebraically closed field $K^{alg}$ for simplicity. However, for any fixed formal group law all reasonings are the same for the fraction field $End(F)\otimes_{\mathbb{Z}_{p}}\mathbb{Q}_{p}$. Also we notice that in this section categories $FGL$ and $\mathcal{M}_{FG}$ are replaceable, because we work with isogeny structures only.

For any fixed formal group law $F(X,Y)$ and any natural number $m$ we consider a category $\mathcal{C}_{F}(m)$ which consists of formal group law $G(X,Y)$ together with a morphism $g:F(X,Y) \rightarrow G(X,Y)$ with a crucial condition
$$Ker [\pi^{m}]_{F} \subset Ker (g).$$
Morphisms in $\mathcal{C}_{F}(m)$ just coincide with morphisms in a coslice category $F \downarrow FGL$.

Then from the Theorem \ref{cor} we know that the object $[\pi^{m}]_{F}: F(X,Y) \rightarrow F(X,Y)$ is initial for $\mathcal{C}_{F}(m)$. So, for instance, for any $(G(X,Y), g)$ from $\mathcal{C}_{F}(m)$ there exists a map $\bar{g}: F(X,Y) \rightarrow G(X,Y)$ such that $g = \bar{g}\circ [\pi^{m}]_{F}$.

\begin{definition}\label{heightfiltration}
We define an endo-analog of $\mathcal{C}_{F}(m)$ as a subcategory\footnote{We identify it with its projection on the second coordinate, thereby identify with subcategory of $End(F)$.} $\mathcal{E}_{F}(m)$ with objects $\left(F(X,Y), f:F(X,Y)\rightarrow F(X,Y)\right)$. Then a natural filtration
$$\mathcal{E}_{F}(0) \subset \mathcal{E}_{F}(1) \subset \mathcal{E}_{F}(2) \subset \dots \;$$
coming from the one on $\{\mathcal{C}_{F}(m)\}_{m}$define \textbf{height filtration} on the ring $End(F)$.
\end{definition}

Now we can formulate and prove the main theorem.

\begin{theorem}
The colimit $\mathcal{E}_{F}(\infty)$ is correctly defined and equivalent to $End(F_{a})$.
\end{theorem}
\begin{proof}
The derivation homomorphism $c: End(F) \rightarrow K^{alg}$ is injective, so we can define a map
$$c^{-1}: Frac(\,c(End(F))\,) \rightarrow End(F)\otimes_{\mathbb{Z}_{p}}\mathbb{Q}_{p}$$
by $c^{-1}(a):= \frac{[a\cdot \pi^{w}]_{F}(x)}{\pi^{w}}$ for $w + v_{L}(a) = 0$. But natural inclusion into $\mathcal{E}_{F}(m)$ sends such $a$ into $\frac{[a\cdot \pi^{m}]_{F}(x)}{\pi^{m}}$. So after an elementary calculation\footnote{The limit is taken in $\pi$-adic topology.}
$$\lim_{m \rightarrow \infty}\frac{F(X,Y) - \left([\pi^{m}]_{F}(X) + [\pi^{m}]_{F}(Y)\right)}{\pi^{m}} = X+Y,$$
we have that all this maps can be continued by $\lim\limits_{m\rightarrow \infty}~\frac{[\pi^{m}\cdot - ]_F}{\pi^m}$~--~operation to maps into the additive group $F_{a}(X,Y)$. Then by the very construction we see that $F_{a}(X,Y)$ is isomorphic  to $\mathcal{E}_{F}(\infty)$.

Finally, we remark that all morphisms are commute with any isomorphisms of formal group $F(X,Y)$, so it is true not only in $FGL$, but also in $\mathcal{M}_{FG}$ .
\end{proof}

The following proposition is a corollary of the proof.

\begin{corollary}
The additive group $F_{a}(X,Y)$ is isomorphic to a dense subgroup in any formal group law $F(X,Y)$.
\end{corollary}
\begin{proof}
From the previous it can be seen that $$F(X,Y)|_{\mathfrak{m}_{K^{alg}}\backslash W_{F}^{\infty}(K^{alg})}\otimes_{\mathbb{Z}_{p}}\mathbb{Q}_{p}$$ is isomorphic to the additive group $F_{a}(X,Y)$ restricted to the same set by the map $\lim\limits_{m\rightarrow} \frac{[\pi^m]_{F}(X)}{\pi^{m}}$.
\end{proof}

So powers of isogenies $[\pi^{m}]_{F}$ can be considered as an analytic approximation sequence for logarithm $log_{F}$ on the punctured set $\mathfrak{m}_{K^{alg}}\backslash W_{F}^{\infty}(K^{alg})$.

Also the last proposition is an incornation of an abstract idea that after localization in the set of all isogenies of any formal group law $F(X,Y)$ we get a representer of $F(X,Y)$ in the $\infty$-component $\mathcal{M}_{FG, \infty}$.

%\textbf{Acknowledgements.}The author was supported by RFBR, grant number 1716--11--10200.


\begin{acknowledgments}
This work was funded by the Russian Science Foundation, grant No
16--11--10200.
\end{acknowledgments}

\begin{thebibliography}{20}
\bibitem{Smith}
            Brian\,D.~Smithling,  On the moduli stack of commutative, 1-parameter formal groups.
             J. of Pure and Applied Algebra \textbf{215} (4), 368--397 (2011).
\bibitem{Tate}
             J. Tate, p-divisible groups. Proc. Dreibergen Summer School on Local Field (Springer, Berlin, 1967).

\bibitem{Lubin1}
             J. Lubin, One-parameter formal Lie groups over p-adic integer rings. Annals of Mathematics
              \textbf{80} (3), 464--484 (1964).
\bibitem{Manin}
             Yu. I. Manin, The theory of commutative formal groups over fields of finite characteristic.
              Russian Math. Surveys \textbf{18} (6), 1--83 (1963).
\bibitem{Lubin}
             J. Lubin, Finite subgroups and isogenies of one-parameter formal Lie groups. Annals of Mathematics
              \textbf{85} (2), 296--302 (1967).



\end{thebibliography}

\end{document}
