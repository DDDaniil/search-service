%%
%% ****** ljmsamp.tex 06.12.2017 ******
%%
\documentclass[
11pt,%
tightenlines,%
twoside,%
onecolumn,%
nofloats,%
nobibnotes,%
nofootinbib,%
superscriptaddress,%
noshowpacs,%
centertags]%
{revtex4}
\usepackage{ljm}

\begin{document}

\titlerunning{CONFIGURATIONS ON  THREE-WEB BOUNDARIES} % for running heads
\authorrunning{A.\,M.\,Shelekhov} % for running heads
%\authorrunning{First-Author, Second-Author} % for running heads

\title{Configurations On  Curvilinear Three-Web Boundaries}
% Splitting into lines is performed by the command \\
% The title is written in accordance with the rules of capitalization.

\author{\firstname{A.\,M.}~\surname{Shelekhov}}
\email[E-mail: ]{amshelekhov@yandex.ru}
\affiliation{Moscow Pedagogical State University, 119991, Moscow, ul. Malaya Pirogovskaya, 1, bdg 1, Russia}




\firstcollaboration{(Submitted by E. K. Lipachev)} % Add if you know submitter.
%%\lastcollaboration{ }

\received{December 7, 2017} % The date of receipt to the editor, i.e. December 06, 2017


\begin{abstract} % You shouldn't use formulas and citations in the abstract.
On the boundaries of the first and second kind of curvilinear three-web, configurations are defined that are analogous to the known Thomsen and Bol configurations. This makes it possible to find the relative invariants defined on the boundaries corresponding to the closure of these configurations.
\end{abstract}

\subclass{53A60, 53A55} % Enter 2010 Mathematics Subject Classification.

\keywords{Curvilinear three-web,  boundaries of a curvilinear three-web, configurations on  a three-web,  differential invariants of a 3-web, regular three-web} % Include keywords separeted by comma.

\maketitle

% Text of article starts here.



\section{Introduction}
\label{intro}

 Let $\lambda_\alpha,  \alpha, \beta, \gamma = 1, 2, 3,$  be 3 smooth families of curves in the plane, $D$ be the maximum possible domain of the plane such that each of its points has a neighborhood in which a) each of the families $ \lambda_\alpha $ forms a foliation; b) families of lines have no singularities and are pairwise transversal. Then we say that the families $\lambda_\alpha$ form a three-web $W$ in the domain $D$, and the domain $D$ is called the domain of definition of the 3-web or the domain of its transversality. Points the families $\lambda_\alpha$ are defined at, but the lines of the families or families themselves have singularities, form the set of singularities of the three-web $W$. In particular, points  the leaves of 3-web $W$ are tangent at do not fall into the domain $D$, and the leaves common to two or three families $\lambda_\alpha$ do not fall. In \cite{SLU} we introduced
  boundary curves of the first kind
$\Gamma_{\alpha\beta}$, at the points of which the lines of the families $\lambda_\alpha$ and $\lambda_\beta$ are tangent. In particular, if the lines of two families coincide, then this common line is called the  boundary curve of the second kind.


The differential topological theory of three-webs studies them  up to local diffeomorphisms preserving only incidence of points or lines of a three-web \cite{B}. Therefore the main object of observation in this theory are configurations formed by three-web lines. Reidemeister, Thomsen, Blaschke, and Bol introduced various configurations ($R, T, H, B$) later generalized to multidimensional three-webs. A necessary and sufficient condition for the closure of all sufficiently small configurations of the same type is equivalent to the vanishing of some relative differential invariant that is a tensor. Thus the classification of three-webs by the closure conditions (and not only
on them) is connected with the finding of relative differential invariants.
\footnote {Absolute invariants are introduced in the framework of the differential topological theory of webs \cite{SLU}, but they do not have a geometric interpretation. Therefore the classification of webs in accordance with absolute invariants is difficult since there is no way to describe the geometric properties of the corresponding classes.}

Note that in terms of relative differential invariants, many important problems in the web theory have been solved, for example, the well-known problem of linearizability of three-webs, see \cite{GL}.


The main relative invariants of a curvilinear three-web are its curvature expressed in terms of derivatives up to the third order of the web function, and the web curvature covariant derivatives \cite{SLU}, \cite{B}.
The geometric meaning of curvature is that it determines the main part of the obstuction to the closure of three-webs configurations.
If the curvature is zero in the domain $D$, then all the classical configurations on the curvilinear three-web are closed. Such a web is equivalent to a web formed by three families of parallel straight lines and is called parallelizable (hexagonal, regular).

But the classical (sufficiently small) configuration of the web and its curvature are defined only in the  domain $D$, that is, outside its boundaries. Meanwhile, the boundaries play an important role in the curvilinear web theory: as shown in [1], a three-web is regular if and only if its boundary curves are lines of this web. Therefore the study of the geometry of webs near the boundaries seems to us an urgent problem.

In the recent paper \cite{AAM}, absolute differential invariants of curvilinear three-webs defined on its boundaries were introduced. In this paper we introduce an analogue of the hexagonal configuration defined at the points of the boundary curve of the first kind, an analogue of the right Bol configuration for the boundary curve of the second kind, and we find the corresponding relative differential invariants, that are, like the curvature, the main part of the obstruction to closure of the introduced configurations. Here, we use a simple but effective technique of computations used by Chern in \cite{Ch}  to find the necessary and sufficient conditions for the closure of classical figures on a multidimensional three-web.

All functions considered in this paper are assumed to be real analytic.

\section{Pentagonal configurations on the boundary of the first kind}

It is known \cite{SLU} that every three-web is equivalent (locally diffeomorphic) to a three-web whose two families form a Cartesian net.
We can always assume that in a neighbourhood of a point in the domain $D$, the third family is the level lines of some function $f(x, y)$. Then
the web equation that connects the parameters of the web lines through a point has the form
$
z=f(x, y).
$
The function $f$ is called the web function.

In Fig. 1 we can see the Thomsen configuration $T$. We are going to construct its analogue.
Here the lines of the first and second families form the Cartesian net, and the lines of the third family
 represented by inclined lines. The vertical, horizontal and slanted lines are marked by the parameters $x_\alpha$, $y_\alpha$ and $z_\alpha$, respectively.  The figure $T$ is constructed, for example, as follows. We take two arbitrary lines $x_1$ and $y_1$ from the first and second families, then we draw the sloping lines $z_1$ and $z_2$ sufficiently close to the point $x_1\cap y_1$. Through the resulting intersection points, we draw the vertical and horizontal lines $x_2$ and $y_2$, $x_3$ and $y_3$, as shown in Fig. 1. We obtain points $A$ and $B$. The resulting configuration is called the Thomsen figure or $T$. If the points
$A$ and $B$ lie on one line of the third family of web, then we say that the figure $T$ is closed.


\begin{figure}[h]
\setcaptionmargin{5mm}
\includegraphics[width=0.45\textwidth]{01.eps}
\captionstyle{normal}
\caption{Thomsen configuration} \label{fig:1}
\end{figure}

In particular, if the lines $x_2$, $y_2$, and $z_2$ pass through a common point, then the Thomsen configuration is called a hexagonal configuration or  figure $H$.
One of the main theorems of the three-web theory is the following: the three-web $ W $ is regular if and only if all sufficiently small figures $ H $ are closed on it.


Since the third family of web lines consists of level lines of the function $f$, the boundary
$\Gamma_{13} $ (at the points the lines of the first and third families are tangent to), is given by the equation
$$
\frac{\partial f}{\partial y}=0.
\eqno (1)
$$


Let $O$ be an arbitrary point on $\Gamma_{13}$. We choose the coordinates so that the point $O$ has zero coordinates. Then in a neighbourhood of this point the curve $\ell$ of the third family  lies on one side of the vertical line $x=0$ that is a line of the first family of web $W$, see Fig. 2. We parametrize the third family so that the equation of the line $\ell$ has the form
$
f(x, y)=0.
$
Let $p(x, 0)$ be an arbitrary point on the line of the second family $y = 0$ that is sufficiently close to $O$ (see Fig. 2). Then the vertical line of the web through $p$ intersects $\ell$ at the points $p_1(x, y_1)$ and $p_2(x, y_2)$. The horizontal web lines through $p_1$ and $p_2$ intersect the $y$ axis at the points $p_3$ and $p_4$, respectively. Let the curves of the third family  through $p_3$ and $p_4$ intersect the axis $x$ at the points $A(x_1, 0)$ and $B(x_2, 0)$, respectively.





\begin{figure}[h]
\setcaptionmargin{5mm}
\includegraphics[width=0.31\textwidth]{1.eps}
\captionstyle{normal}
\caption{Pentagonal configuration $P_{13}$} \label{fig:2}
\end{figure}






The constructed pentagonal configuration is denoted by $P_{13}$. It can be regarded as an analogue of the configuration $H$ for the boundary
$\Gamma_{13}$.

We estimate the difference $x_2-x_1$ up to the fourth order. First
of all, note that, in view of (1)
$$
\frac{\partial f}{\partial y}(0,0)=0.
$$
Therefore in a neighbourhood of the point $O$, we have the expansion
$$
\aligned &f(x,
y)=f_x(0,0)x+\frac{1}{2}f_{xx}(0,0)x^2+f_{xy}(0,0)xy+f_{yy}(0,0)y^2
\\&+\frac{1}{6}f_{xxx}(0,0)x^3+\frac{1}{2}f_{xxy}(0,0)x^2y+\frac{1}{2}f_{xyy}(0,0)xy^2+\frac{1}{6}f_{yyy}(0,0)y^3+O_4.
\endaligned
\eqno (2)
$$
Since the points $O$ and $p_1$ lie on the line $\ell$, we have $f(0, 0)=f(x, y_1)$ or
$$
\aligned &0\sim
f_x(0,0)x+\frac{1}{2}f_{xx}(0,0)x^2+f_{xy}(0,0)xy_1+f_{yy}(0,0)(y_1)^2
\\&+\frac{1}{6}f_{xxx}(0,0)x^3+\frac{1}{2}f_{xxy}(0,0)x^2y_1+\frac{1}{2}f_{xyy}(0,0)x(y_1)^2+\frac{1}{6}f_{yyy}(0,0)(y_1)^3.
\endaligned
\eqno (3)
$$
The symbol $\sim$ means that the calculations are carried  out up to
the fourth order terms. In what follows, we write $f_x$ instead of
$f_x (0,0)$ and so on.

It is clear from (3) that we can put
$$
x \sim -\frac{f_{yy}}{2f_x} (y_1)^2+v(y_1)^3,
\eqno (4)
$$
Substituting into (3) we find:
$$
v\sim\frac{f_{xy}f_{yy}}{2(f_x)^2}-\frac{f_{yyy}}{6f_x}.
\eqno (5)
$$
Similar conclusions are obtained for the point $p_2$:
$$
x \sim -\frac{f_{yy}}{2f_x} (y_2)^2+v(y_2)^3.
\eqno (6)
$$

Let $ f (x, y) = c $ be the line of the third family through the points $ p_3 (0, y_1) $
and $ A (x_1, 0) $, then $ f (0, y_1) = f (x_1, 0) $, or, by virtue of (2),
$$
\frac{1}{2}f_{yy}(y_1)^2+\frac{1}{6}f_{yyy}(y_1)^3\sim f_x x_1+\frac{1}{2}f_{xx}(x_1)^2+\frac{1}{6}f_{xxx}(x_1)^3.
$$
Hence we get
$
x_1\sim \frac{1}{f_x}(\frac{1}{2}f_{yy}(y_1)^2+\frac{1}{6}f_{yyy}(y_1)^3).
$
Similarly we find
$
x_2\sim \frac{1}{f_x}(\frac{1}{2}f_{yy}(y_2)^2+\frac{1}{6}f_{yyy}(y_2)^3).
$
From the last two equalities we have:
$$
x_2-x_1\sim \frac{f_{yy}}{2f_x}((y_2)^2-(y_1)^2)+\frac{f_{yyy}}{6f_x}((y_2)^3-(y_1)^3).
\eqno (7)
$$

On the other hand, from (4) and (6) we find
$
\frac{f_{yy}}{2f_x}((y_2)^2-(y_1)^2)\sim v((y_2)^3-(y_1)^3).
$
Substituting into (7), taking into account (5) we obtain:
$$
x_2-x_1\sim (v+\frac{f_{yyy}}{6f_x})((y_2)^3-(y_1)^3)=\frac{f_{xy}f_{yy}}{2(f_x)^2}((y_2)^3-(y_1)^3).
$$
Thus the main part of the difference $ x_2-x_1 $ is of order 3 and is determined by the quantity
$$
\pi_{13}=\frac{f_{xy}f_{yy}}{(f_x)^2},
\eqno (8)
$$
where the values of the derivatives are calculated at an arbitrary
point of  the boundary $ \Gamma_{13} $. It follows that $\pi_{13}$
is a relative invariant of the three-web $W$ on the boundary of the
first kind $\Gamma_{13}$.



We proved

\begin{theorem}\label{Th:1}
Let the three-web $W$ be formed by the Cartesian net and the level
lines of the function $f$. Then the main part of the obstruction to
the closure  of the figure $P_{13}$ constructed on the boundary of
the first kind $\Gamma_{13}$ is determined by the relative invariant
$\pi_{13}$ computed at the points of this boundary by the formula
(8).
\end{theorem}

The similar relative invariant on the boundary $\Gamma_{23}$ will be
$$
\pi_{23}=\frac{f_{xy}f_{xx}}{(f_y)^2}.
$$



\section{Analogue of Bol configuration on the boundary of the second kind}
Let the line $ x = 0 $ be the common line of the first  and third
families, that is, the line is the boundary $ \Gamma_{13} $ of the
second kind of the 3-web $W$, given by the Cartesian net and some
family of lines $F(x, y, c)=0$. Let the line $ x = 0 $ have the
parameter $ c = 0 $, then
$$
F(x, y, c)=x+a(x, y)c+b(x, y)c^2+\ldots.
\eqno (9)
$$


We place the origin  in an arbitrary point of the axis  $ y $ and
construct Fig. 3. We choose an arbitrary point $ p (x, 0) $, draw a
line of the third family through it, take an arbitrary point $ p_1
(x_1, y) $ on the latter, and take $ x $ and $ y $ close to zero.
Next, on axis $ x $ we take the point $ \bar p (-x, 0) $, draw the
line of the third family through it, take the point $ p_2 (x_2, y) $
on it. Through the points $ p_1 $ and $ p_2 $ we draw the vertical
lines of the web $W$, then we obtain the points $ p_3 (x_1, 0) $ and
$ p_4 (x_2, 0) $, respectively. Through the points $ p $ and $ \bar
p $ we draw vertical lines of the web $ W $, through the points $
p_3 $ and $ p_4 $ draw inclined lines. We obtain the points $ p_5
(x, y_1) $ and $ p_6 (-x, y_2) $, respectively.

The constructed configuration coincides with the well-known  right
Bol figure $ B_r $ [1]. The fundamental difference between this
figure and the classical one is that in the latter the left and
right parts are not separated by the boundary,
 that is, they lie in one connected part of the domain $D$ of the 3-web $W$. The configuration in Fig. 3 we denote by $ \mathcal {B}_{13} $.
If the constructed configuration is closed, then $ y_1 = y_2 $. Therefore the main part of the difference $ y_1-y_2 $ is a relative invariant of the 3-web $W$.


\begin{figure}[h]
\setcaptionmargin{5mm}
\includegraphics[width=0.45\textwidth]{2.eps}
\captionstyle{normal}
\caption{Analogue of the Bol configuration $\mathcal{B}_{13}$} \label{fig:3}
\end{figure}


We compute $ y_1 $ and $ y_2 $, assuming the figure $ \mathcal
{B}_{13} $ to be sufficiently small.  This means that the quantities
$ x $, $ y $  defining this figure, and the parameters of the lines
of the third family entering into it, should be assumed to be close
to zero.

We write down the conditions for the pairs of points $ p $ and $ p_1
$, $ p_3 $ and $ p_5 $, $ \bar p $ and $ p_2 $, $ p_4 $ and $ p_6 $
 to belong to the same curve of the third family:
$$
\aligned
(1)\quad & F(x, 0, c_1)=0, \quad F(x_1, y, c_1)=0;\\
(2)\quad & F(-x, 0, c_2)=0, \quad F(x_2, y, c_2)=0;\\
(3)\quad & F(x_1, 0, c_3)=0, \quad F(x, y_1, c_3)=0;\\
(4)\quad & F(x_2, 0, c_4)=0, \quad F(-x, y_2, c_4)=0.
\endaligned
\eqno (10)
$$

It is clear from (9) that
$$
F_c(x, y, 0)=a(x, y), \, F_x(x, y, 0)=1,
$$
all the other derivatives of $ F $ with respect to  the variables $
x $ and $ y $ for $ c = 0 $ (that is, on the boundary $ \Gamma_{13}
$) being equal to zero.


Let the series expansion of $ a (x, y) $ be written as
$$
a(x, y)=a_0+a_1x+a_2y+a_{11}x^2+a_{12}xy+a_{22}y^2+a_{111}x^3+a_{112}x^2y+a_{122}xy^2+a_{222}y^3+\ldots.
\eqno (11)
$$
We assume that $ a_0 \neq 0 $, that is, the quantities $ x $ and $ y $ are small relative to
$ a(x, y) $. Then, up to the second order members, we obtain from (9) and (10, 1):
$
x+a(x, 0)c_1\sim 0, \quad x_1+a(x_1, y)c_1\sim 0,
$
and
$$
x a(x_1, y)\sim x_1a(x, 0).
\eqno (12)
$$
We look for $ x_1 $ in the form
$
x_1=u_0+u_1x+u_2y+u_{11}x^2+\ldots
$
Substituting in the previous equation and using (11), after some calculations we get the following formula.


\begin{lemma}\label{L:1}
$$
x_1\sim x+\frac{a_2}{a_0}xy+\frac{a_{12}}{a_0}x^2y+\frac{a_{22}}{a_0}xy^2+(\frac{a_{112}}{a_0}+\frac{a_2a_{11}}{a_0^2})x^3y+(\frac{a_{122}}{a_0}+\frac{a_2a_{12}}{a_0^2})x^2y^2+\frac{a_{222}}{a_0}xy^3.
\eqno (13)
$$
\end{lemma}

From (10.2) we similarly find
$
-x a(x_2, y)\sim x_2a(-x, 0).
$
This equality is obtained from (17) by replacing the variable $ x $ by $ -x $, so the following is true.

\begin{lemma}\label{L:2}
$$
x_2\sim -x-\frac{a_2}{a_0}xy+\frac{a_{12}}{a_0}x^2y-\frac{a_{22}}{a_0}xy^2-(\frac{a_{112}}{a_0}+\frac{a_2a_{11}}{a_0^2})x^3y+(\frac{a_{122}}{a_0}+\frac{a_2a_{12}}{a_0^2})x^2y^2-\frac{a_{222}}{a_0}xy^3.
$$
\end{lemma}

The third equality (10) gives
$
x a(x_1,0)\sim x_1a(x, y_1).
$
Substituting here $ x_1 $ from (18) and solving with respect to $ y_1 $, we obtain the following result.

\begin{lemma}\label{L:3}
$$
y_1\sim -y+u_{22}y^2+u_{122}xy^2-u_{22}^2y^3,
$$
where
$
u_{22}=\frac{a_2}{a_0}-2\frac{a_{22}}{a_2}, \quad
u_{122}=-2\frac{a_{122}}{a_2}+2\frac{a_{12}a_{22}}{a_2^2}.
$
\end{lemma}
Consider the fourth equality (10). It is obtained  from the second
one by replacing $ x $ by $ -x $ in the right-hand side and by
changing $ x_1 $ to $ x_2 $ in the left-hand side. But $ x_2 $ is
obtained from $ x_1 $ also by replacing $ x $ by $ -x $. Therefore $
y_2 $ is obtained from $ y_1 $ also by replacing $ x $ by $ -x $,
that is, the following is true.

\begin{lemma}\label{L:4}
$
y_2\sim -y+u_{22}y^2-u_{122}xy^2-u_{22}^2y^3.
$
\end{lemma}


From the above we get
$
y_1-y_2\sim 2u_{122}xy^2\equiv2\beta_{13}xy^2.
$
So, the quantity
$$
\beta_{13}=-2\frac{a_{122}}{a_2}+2\frac{a_{12}a_{22}}{a_2^2}
\eqno (14)
$$
is a relative invariant of the three-web on the 2-kind boundary $ \Gamma_{13} $. We have proved

\begin{theorem}\label{Th:2} Let the
 three-web be formed by the Cartesian net and family
 of curves (9), so that the $ y $ axis is the boundary
  curve of the second kind $ \Gamma_{13} $.
Then the main part of the obstruction to the closure of
 the figure $ \mathcal {B}_{13} $ constructed on the
  boundary $\Gamma_{13}$ is determined by the relative invariant $\beta_{13} $
  calculated at every point of this boundary by formula (14).
\end{theorem}

\section{Calculation of the invariant $\beta_{13}$}

We express $ \beta_{13} $ in terms of partial derivatives of the function $ a $. From (11) we obtain:
$$
a_2=a_y(0, 0), a_{12}=a_{xy}(0, 0), 2a_{22}=a_{yy}(0, 0), 2a_{122}=a_{xyy}(0, 0).
$$
Substituting in the second formula (14), we obtain:
$$
\beta_{13}=-\frac{a_ya_{xyy}-a_{xy}a_{yy}}{a_y^2}=-(\frac{a_{xy}}{a_{y}})_y=-(\ln a_y)_{xy},
$$
all derivatives being computed at point $ (0, 0) $, and the last
equality is possible in a domain, where $ a_y> 0 $. But since $ (0,
0) $ is an arbitrary point of axis $ y $, we can write
$$
\beta_{13}=-(\ln a_y)_{xy}|_{c=0}.
\eqno (15)
$$

On the other hand, it follows from (9) that for $ c = 0 $ $ F_c = a (x, y), $
$$
 F_{cy}=a(x, y)_y, \, F_{cxy}=a(x, y)_{xy}, \,F_{cyy}=a(x, y)_{yy}, \,F_{cxyy}=a(x, y)_{xyy}.
$$
Therefore for $ c = 0 $
$
(\ln a_y)_{xy}=(\ln F_{cy})_{xy}
$
and (15) takes the form
 $$
\beta_{13}=-(\ln F_{cy})_{xy}|_{c=0}.
\eqno (16)
$$


Let us find the form of the invariant $ \beta_{13} $ for the case
when the third family is given by the level lines of the function $
f (x, y) $. Then the identity $ F (x, y, f (x, y)) \equiv 0 $ is
fulfilled. Differentiating it we obtain $ F_x+F_cf_x=0. $ Since $ F_
{x} = 1 $, $ F_ {c} = a $ for $ c = 0 $, from the previous formulas
we have  $1+af_x=0$ for $ c = 0 $.  As a result formula (15) takes
the form:
$$
\beta_{13}=(\ln (\frac{1}{f_x})_y)_{xy}|_{c=0}.
$$


For the boundary $\Gamma_{23} $ of the second kind,  we can
construct an analogous figure $ \mathcal {B}_{23} $ and find
analogous formulas for the corresponding relative invariant.

\section{The necessary condition for the $ P_{13} $ closure}

The necessary condition for the closure of the figure $ P_{13} $ is that the relative invariant
$ \pi_{13} $ vanishes, which gives the equality
$
f_{xy}f_{yy}=0
$
that must be satisfied at the points of the boundary of the first kind $\Gamma_{13}$.

1) Consider the case
$$
f_{xy}=0.
\eqno (17)
$$

If condition (17) is satisfied at every point of the web domain, then
$ f = \alpha (x) + \beta (y) $ and the web is regular [2].

If condition (17) is satisfied on the boundary $ \Gamma_{13} $
(given by the equation $ f_y = 0 $), then the condition
$$
f_{xy}=\Theta(f_y, x, y),\quad \Theta(0, x, y)=0
\eqno (18)
$$
must be fulfilled.

Solutions of equation (18) can be obtained as follows. We set $
f_y=(p(x, y))^k, \quad k\neq 0, 1. $ Then $
f_{xy}=kp^{k-1}p_x=k(f_y)^\frac{k-1}{k}p_x, $ and  condition (18) is
satisfied. In this case, the boundary  $ \Gamma_ {13} $ is given by
the equation $ p = 0 $, and the function $ f $ can be found by
integration. The three-web $ W $ in this case will not, in general,
be regular.

2) The case
$$
f_{yy}=0.
\eqno (19)
$$

If condition (19) is satisfied at every point of the web domain $D$,
then the web equation has the form $ z = a (x) y + b (x) $, and in
the plane of variables $ y, z $ it defines a rectilinear three-web $
\bar W $ obtained from the web $ W $ by renumbering foliations and
formed by the Cartesian net $ y = const $, $ z = const $ and the
family of straight lines $ z = a (x) y + b (x) $ with parameter $ x
$.

If (19) is satisfied on the boundary $ \Gamma_{13} $, then the condition
$
f_{yy}=\Theta(f_y, x, y),\quad \Theta(0, x, y)=0
$
must be fulfilled. Solutions of this equation can be found similarly to 1).


\section{The necessary condition for the $ \mathcal {B}_{13} $ closure}
The necessary condition for the closure of the figure $ \mathcal
{B}_{13} $  is that the relative invariant $ \beta_{13} $ on the
boundary $ \Gamma_{13} $ (its equation is $ x = 0 $) is equal to
zero. Suppose that the web is given by equation (9), then we obtain
from (16)
 $$
(\ln F_{cy})_{xy}|_{c=0}=0.
\eqno (20)
$$

If the equality $ (\ln F_{cy})_{xy} = 0 $ is satisfied at every point of domain $D$, then
after integration we obtain
 $$
F=x+c(\alpha(x)\beta(y)+\gamma(x)),
\eqno (21)
$$
where $\alpha(x)\beta(y)\neq 0$ in $D$ (otherwise  the web does not
exist). The three-web, defined in this case by the equation $ F = 0
$, has not only the boundary $ \Gamma_{13} $ of the second kind ($ x
= 0 $) but also the boundary $ \Gamma_{13} $  of the first kind
defined by the equation $ \beta(y) = 0 $. Outside the boundary $ x =
0 $, the web equation $ F = 0 $ can be written in the form $
c^{-1}+x^{-1}(\alpha(x)\beta(y)+\gamma(x))=0, $ or, after admissible
substitutions of variables  $c^{-1}\rightarrow z,\, x^{-1}\alpha(x)
\rightarrow x,\, \beta(y)\rightarrow -y$, in the form $
z=xy+\varphi(x). $ Generally speaking this web is not regular and
will be  such if $ \varphi (x) $ is a linear function. Note that the
last equation
 determines a rectilinear three-web in the plane of the variables y and z.

In accordance with (21), the solution of equation (20) can be written in the form
$$
F=x+c(\alpha(x)\beta(y)+\gamma(x))+\varphi_2(x, y)c^2+\varphi_3(x, y)c^3+\ldots.
$$


\section{On the sufficient condition for the $ \mathcal {B}_{13} $ closure}

 The necessary closure conditions for the figures $ P_{13} $ and $ \mathcal {B}_{13} $ are not, generally speaking, sufficient. Let us find, for example, a sufficient condition for the closure of the figures $ \mathcal {B}_{13} $ for the web (21).  First we make an admissible  substitution
 $ \beta (y) \rightarrow y $, then the web equation  takes the form
$$
x+c(\alpha(x)y+\gamma(x))=0,
\eqno (21')
$$
and the closure conditions (10) are written in the form:
$$
\aligned
(1)\quad & x+\gamma(x)c_1=0, \quad x_1+(\alpha(x_1)y+\gamma(x_1))c_1=0;\\
(2)\quad & -x+\gamma(-x)c_2=0, \quad x_2+(\alpha(x_2)y+\gamma(x_2))c_2=0;\\
(3)\quad & x_1+\gamma(x_1)c_3=0, \quad x+(\alpha(x)y_1+\gamma(x))c_3=0;\\
(4)\quad & x_2+\gamma(x_2)c_4=0, \quad -x+(\alpha(-x)y_2+\gamma(-x))c_4=0.
\endaligned
$$
Since the configuration lines are outside the domain $ c = 0 \, (x = 0) $, we find from these relations:
$$
\aligned
(1)\quad & \frac{\gamma(x)}{x}=\frac{\alpha(x_1)}{x_1}y+\frac{\gamma(x_1)}{x_1};\\
(2)\quad & \frac{\gamma(-x)}{-x}=\frac{\alpha(x_2)}{x_2}y+\frac{\gamma(x_2)}{x_2};\\
(3)\quad & \frac{\gamma(x_1)}{x_1}=\frac{\alpha(x)}{x}y_1+\frac{\gamma(x)}{x};\\
(4)\quad & \frac{\gamma(x_2)}{x_2}=\frac{\alpha(-x)}{-x}y_2+\frac{\gamma(-x)}{-x}.
\endaligned
\eqno (22)
$$
Adding the first and the third equalities, then the second to the fourth ones we obtain equalities
$$
\frac{\alpha(x_1)}{x_1}y+\frac{\alpha(x)}{x}y_1=0;\quad \frac{\alpha(x_2)}{x_2}y+\frac{\alpha(-x)}{-x}y_2=0.
$$
A necessary and sufficient condition for the closure $ y_1 = y_2 $ is satisfied in the case when
$$
\frac{\alpha(x_1)}{x_1}:\frac{\alpha(x)}{x}=\frac{\alpha(x_2)}{x_2}:\frac{\alpha(-x)}{-x},\quad
{\rm or}\quad
\frac{\alpha(x_1)}{x_1\alpha(x)}=-\frac{\alpha(x_2)}{x_2\alpha(-x)}.
$$
The values of $ x_1 $ and $ x_2 $ can be found from the first two equations (22): $x_1=\chi(x, y)$,  $x_2=\chi(-x, y)$, then the previous equality takes the form:
$$
\frac{\alpha(\chi(x, y))}{\chi(x, y)\alpha(x)}=-\frac{\alpha(\chi(-x, y))}{\chi(-x, y)\alpha(-x)}.
$$
So \textit{the necessary and sufficient closure condition is that the function $ \alpha \circ \chi / \chi \cdot \alpha $ be skew-symmetric.}

Remark. The construction of  figures $ P_{13} $ and $ \mathcal
{B}_{13} $ is  impossible if the web exists only on one side of the
boundary,  for example, if three-web is given by the equation
$(z-x)^2+y^2=1$.

\begin{thebibliography}{99}

\bibitem[1]{SLU}
A.~M. Shelekhov, V.~B. Lazareva  and A.~A. Utkin, \textit{Curvilinear three-webs} (Tver State University, Tver, 2013) [In Russian].

\bibitem[2]{B} W. Blaschke, \textit{Einf\"{u}hrung in die Geometrie der Waben}
(Birkh\"{a}user-Verlag, Basel-Stuttgart, 1955).

\bibitem[3]{GL} V.~V. Goldberg  and V.~V. Lychagin,
\textit{On the Blaschke conjecture for $3$-webs}, J. of Geometric
Analysis. Suvreys \textbf{16} (1), 69--115 (2006).

\bibitem[4]{AAM}
 F.~A. Arias Amaya, J.~R. Arteaga Bejarano and M. Malakhaltsev, \textit{3-webs with singularities}, Lobachevskii J. of Mathematics. Suvreys  \textbf{37}, (1), 1--20 (2016).

\bibitem[5]{Ch}
S.~S. Chern,
\textit{Eine Invariantentheorie der Dreigewebe aus
$r$-dimensionalen Mannigfaltigkeiten in ${R}_{2r}$}, Abh.
Math.  Sem. Univ. Hamburg. Suvreys \textbf{11} (1--2), 333--358 (1936).

\end{thebibliography}
\end{document}
