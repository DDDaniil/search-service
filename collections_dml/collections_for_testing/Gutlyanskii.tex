%%
%% ****** ljmsamp.tex 13.06.2018 ******
%%
\documentclass[
11pt,%
tightenlines,%
twoside,%
onecolumn,%
nofloats,%
nobibnotes,%
nofootinbib,%
superscriptaddress,%
noshowpacs,%
centertags]%
{revtex4}
\usepackage{ljm}
\begin{document}

\titlerunning{Quasiconformal mappings in the theory  of semi-linear
equations} % for running heads
%\authorrunning{V. Gutlyanski\u\i{},  V. Ryazanov} % for running heads
\authorrunning{V. Gutlyanski\u\i{},  V. Ryazanov} % for running heads

\title{Quasiconformal Mappings in the Theory of Semi-linear
Equations}
% Splitting into lines is performed by the command \\
% The title is written in accordance with the rules of capitalization.

\author{\firstname{V.}~\surname{Gutlyanski\u\i{}}}
\email[E-mail: ]{vgutlyanskii@gmail.com}
%\affiliation{Institute of Applied Mathematics and Mechanics, National Academy of Sciences of Ukraine, Dobrovolsky St. 1, Sloviansk, 84100, Ukraine}
%\affiliation{Example: N.I. Lobachevskii Institute of Mathematics and Mechanics, Kazan (Volga Region) Federal University, Kremlevskaya ul. 18, Kazan, Tatarstan, 420008 Russia}

\author{\firstname{V.}~\surname{Ryazanov}}
\email[E-mail: ]{vl.ryazanov1@gmail.com}
\affiliation{Institute of Applied Mathematics and Mechanics of National Academy of Sciences of Ukraine, Dobrovolsky ul. 1, Sloviansk, 84100  Ukraine}
%\affiliation{Place of work and/or the address of second authors}
%\noaffiliation % If the author does not specify a place of work.

\firstcollaboration{(Submitted by A.~M.~Elizarov)} % Add if you know submitter.
%\lastcollaboration{ }

\received{June 13, 2018} % The date of receipt to the editor, i.e. December 06, 2017


\begin{abstract} % You shouldn't use formulas and citations in the abstract.
We study the Dirichlet problem with continuous boundary data in
simply connected domains D of the complex plane for the semi--linear
partial differential equations whose linear part has the divergent
form. We prove that if a Jordan domain D satisfies the so-called
quasihyperbolic boundary condition, then the problem has regular
(continuous) weak solutions whose first generalized derivatives by
Sobolev are integrable in the second degree. We give a suitable
example of a Jordan domain that fails to satisfy both the
well--known (A)--condition and the outer cone condition. We also
extend these results to some non-Jordan domains in terms of the
prime ends by Caratheodory. The proofs are based on our
factorization theorem established earlier. This theorem allows us to
represent solutions of the semi-linear equations in the form of
composition of solutions of the corresponding quasilinear Poisson
equation in the unit disk and quasiconformal mapping of D onto the
unit disk generated by the measurable matrix function of
coefficients. In the end we give applications to relevant problems
of mathematical physics in anisotropic inhomogeneous media.
\end{abstract}

\subclass{Primary 30C62,
31A05, 31A20, 31A25, 31B25, 35J61 Secondary 30E25, 31C05, 34M50,
35F45, 35Q15} % Enter 2010 Mathematics Subject Classification.

\keywords{conformal and quasiconformal mappings,
semi-linear elliptic equations, anisotropic and inhomogeneous
media} % Include keywords separeted by comma.

\maketitle

% Text of article starts here.

\section{Introduction} Given a domain $D$ in ${\Bbb C},$ denote by $M^{2\times 2}_K(D)$ the
class of all $2\times 2$ symmetric matrix function
$A(z)=\{a_{jk}(z)\}$ with measurable entries and ${\rm
det}\,A(z)=1,$ satisfying the uniform ellipticity condition
\begin{equation} \label{eqM}
{1\over K}\ |\xi|^2\ \leq\ \langle\,
A(z)\, \xi,\,\xi\, \rangle\ \leq\ K\, |\xi|^2 \,\,\, \,\,\, \,\,\,
\mbox{a.e. in}\,\, D
\end{equation} for every $\xi\in{\Bbb C}$,
where $1\leq K<\infty$. Further we study the semi-linear equations
\begin{equation} \label{eqQUASI} {\rm div\,}[\, A(z)\, \nabla u(z)\, ]\ =\
f(u(z)), \,\,\,z\in D,
\end{equation}
with continuous functions $f:\mathbb R\to\mathbb R$
either bounded or such that    $f(t)/t\to 0$ as $t\to\infty.$
Simi-linear equations with such $f$  describe a number of physical
phenomena in anisotropic inhomogeneous media. The equations
(\ref{eqQUASI}) are closely relevant to the so-called Beltrami
equations. Let $\mu: D\to\mathbb C$ be a measurable function with
$|\mu(z)|<1$ a.e. The equation
\begin{equation}\label{1}
\omega_{\bar{z}}=\mu(z)~\omega_z,
\end{equation}
where $\omega_{\bar z}=(\omega_x+i\omega_y)/2$,
$\omega_{z}=(\omega_x-i\omega_y)/2$, $z=x+iy$, $\omega_x$ and
$\omega_y$ are partial derivatives of the function $\omega$ in $x$
and $y$, respectively, is said to be a {\bf Beltrami equation}.
The equation~\eqref{1} is said to be {\bf nondegenerate} if
$||\mu||_{\infty}<1$. The homeomorphic solutions of nondegenerate
Beltrami's equations~\eqref{1} in $W^{1,2}_{\rm loc}$ are called
{\bf quasiconformal mappings}, see e.g. \cite{Ahlfors:book, AIM} and \cite{LV:book}.

We say that a quasiconformal mapping $\omega$ satisfying (\ref{1})
is {\bf agreed with} $A\in M^{2\times 2}_K(D)$ if
\begin{equation}\label{mu} \mu(z)\ =\ \frac{a_{22}(z)-a_{11}(z)-2i
a_{12}(z)}{{\rm det}\,\,(I+A(z))},
\end{equation}
where $I$ is the unit $2\times 2$ matrix. Condition (\ref{eqM})
is now written as
 \begin{equation}\label{ellipticity1}
 |\mu(z)|\
\leq\ (K-1)/(K+1) \,\,\,\,\,\mbox{a.e. in $D$}.
\end{equation}
Vice
versa, given a measurable function $\mu: D\to\mathbb C$, satisfying
(\ref{ellipticity1}), one can invert the algebraic system (\ref{mu})
to obtain the matrix function $A\in M^{2\times 2}_K(D)$:
\begin{equation}\label{matrix}
A(z)\ =\ \left(\begin{array}{ccc} {|1-\mu|^2\over 1-|\mu|^2}  & {-2{\rm Im}\,\mu\over 1-|\mu|^2} \\
                            {-2{\rm Im}\,\mu\over 1-|\mu|^2}          & {|1+\mu|^2\over 1-|\mu|^2}
                              \end{array}\right).
 \end{equation}
By the existence theorem for (\ref{1}), see e.g. Theorem V.B.3 in
\cite{Ahlfors:book} and Theorem V.1.3 in \cite{LV:book}, any $A\in
M^{2\times 2}_K(D)$ generates a quasiconfomal mapping
$\omega:D\to\mathbb D$.

We also would like to pay attention to a strong interaction between
linear and non-linear elliptic systems in the plane and
quasiconformal mappings. The most general first order linear
homogeneous elliptic system with real coefficients can be written in
the form $f_{\bar z}+\mu(z)f_z +\nu(z) \overline{f_z}=0,$ with
measurable coefficients $\mu$ and $\nu$ such that $|\mu|+|\nu|\leq
(K-1)/(K+1)<1.$ This equation is a particular case of a non-linear
first order system $f_{\bar z}=H(z,f_z)$ where $H:G\times{\Bbb
C}\to{\Bbb C}$ is Lipschitz in the second variable,
$$|H(z,w_1)-H(z,w_2)|\leq {K-1\over K+1}|w_1-w_2|,\,\,\,\, H(z,0)\equiv 0.$$
The principal feature of the above equation is that the difference
of two solutions  need not solve the same equation but each solution
can be represented as {\it a composition of a quasiconformal
homeomorphism and an analytic function.} Thus quasiconformal
mappings become the central tool for the study of these non-linear
systems. A rather comprehensive treatment of the present state of
the theory is given in the excellent book of Astala, Iwaniec and
Martin \cite{AIM}. This book contains also an exhaustive
bibliography on the topic.
 In particular, the following fundamental Harmonic
Factorization Theorem for the uniformly elliptic divergence
equations \begin{equation}\label{de} {\rm div\,}A(z,\nabla
u)=0,\,\,\,z\in\Omega, \end{equation} holds, see \cite{AIM}, Theorem
16.2.1: Every solution $u\in W^{1,2}_{\rm loc}(\Omega)$
 of the  equation (\ref{de})
can be expressed as {\it the composition  $u(z) = h(f(z))$ of a
quasiconformal homeomorphism $f : \Omega\to G$ and a suitable
harmonic function $h$ on $G.$}

The main goal of this paper is  to point out another application of
quasiconformal mappings to the study of some  semi-linear partial
differential equations, linear part of which contains the elliptic
operator in the divergence form ${\rm div}\,[A(z)\nabla u(z)].$

A fundamental role in the study of the posed problem will play
  Theorem 4.1 in \cite{GNR2017}, that can be considered as a suitable counterpart to the mentioned
above Factorization theorem: a function $u:D\to\mathbb R$ is a weak
solution of (\ref{eqQUASI}) in the class $C\cap W^{1,2}_{\rm
loc}(D)$ if and only if $u=U\circ\omega$ where $\omega :D\to\mathbb
D$ is a quasiconformal mapping agreed with $A$ and $U$ is a weak
solution in the class $C\cap W^{1,2}_{\rm loc}(\mathbb D)$ of the
quasilinear Poisson equation
\begin{equation}\label{QUASII}
\triangle\, U(w)\ =\ J(w)~f(U(w)),\ \ \ \ w\in\mathbb D,
\end{equation}
where $J$ denotes the Jacobian of the inverse quasiconformal mapping
$\omega^{-1}:\mathbb D\to D$. Here a {\bf weak solution} to
(\ref{eqQUASI}) is a function $u\in C\cap W^{1,2}_{\rm loc}(\Omega)$
such that $$\int\limits_D \langle A(z)\nabla
u(z),\nabla\eta(z)\rangle\ dm(z) + \int\limits_D f(u(z))\, \eta(z)\
dm(z) = 0\ \ \ \forall\  \eta\in C\cap W^{1,2}_0(D),$$ where $m(z)$
stands for the Lebesgue measure in the plane.


\section{Definitions and preliminary remarks}

In the  paper \cite{GNR2018}, Theorem 3, we have established the
following statement on the existence of regular weak solutions of
the Dirichlet problem for a quasilinear Poisson equation.


{\bf Proposition 1.} {\it Let $\varphi :\partial\mathbb D\to\mathbb
R$ be a continuous function, $h:\mathbb D\to\mathbb R$ be a function
in the class $L^p(\mathbb D)$, $p>1$,  and let $f:\mathbb
R\to\mathbb R$ be a continuous function that is either bounded or
with the nondecreasing function $|f\,|$ of $\ |t|$ such that
\begin{equation} \label{eqAPRIORY}
\lim\limits_{t\to +\infty}\ \frac{f(t)}{t}\ =\ 0.
\end{equation}
Then there exist weak solutions $U$ of the quasilinear Poisson
equation
\begin{equation} \label{eqQUASILINEAR}
\triangle\, U(z)\ =\ h(z)~ f(U(z))
\end{equation}
such that $U\in C(\overline{\mathbb D}),$
$U|_{\partial\mathbb D}\equiv\varphi.$ More precisely, $U|_{\mathbb
D}\in W^{2,p}_{\rm loc}(\mathbb D)$ and (\ref{eqQUASILINEAR}) holds
a.e. in ${\mathbb D}.$ Moreover, $U\in W^{1,q}_{\rm loc}(\mathbb D)$
for some $q>2$ and $U$ is locally H\"older continuous in $\mathbb
D$. If in addition $\varphi$ is H\"older continuous, then $U$ is
H\"older continuous in $\overline{\mathbb D}$. If $p>2$, then $U\in
C^{1,\alpha}_{\rm loc}(\mathbb D)$ where $\alpha = (p-2)/p$. In
particular, $U\in C^{1,\alpha}_{\rm loc}(\mathbb D)$ for all
$\alpha\in(0,1)$ if $h\in L^{\infty}(\mathbb D)$.}


%\bigskip

Thus, the degree of regularity of the weak solutions of the
Dirichlet problem to (\ref{eqQUASILINEAR}) essentially depends on
the degree of integrability of the multiplier $h$. Furthermore, by
an example in \cite{GNR2018} the equation (\ref{eqQUASILINEAR}) can
have no continuous solutions if $h$ is only in the class
$L^1(\mathbb D)$.

%\bigskip

Making use of fundamental results on boundary correspondence under
conformal and quasiconformal mappings, one can extend the above
statement replacing the unit disk ${\mathbb D}$ by a smooth
Jordan's domain.

%\medskip

 {\bf Corollary 1.} {\it Let $D$ be a smooth $(C^1)$ Jordan's
domain in $\mathbb C$, $\Phi :\partial D\to\mathbb R$ be a
continuous function, $H: D\to\mathbb R$ be a function in the class
$L^p(D)$, $p>1$, and let $f:\mathbb R\to\mathbb R$ be a continuous
function which is either bounded or satisfying (\ref{eqAPRIORY})
with nondecreasing $|f\,|$ of $\ |t|$. Then there exist weak
solutions $u$ of the quasilinear Poisson equation
\begin{equation} \label{eqQUASILINEARC}
\triangle\, u(\zeta)\ =\ H(\zeta)~ f(u(\zeta))\ \ \ \ \ \ \
\mbox{for a.e.}\ \zeta\in D\
\end{equation}
such that $u\in C(\overline{ D}),$ $u|_{\partial D}\equiv\varphi.$
More precisely, $u|_{\mathbb D}\in W^{2,p}_{\rm loc}(\mathbb D)$ and
(\ref{eqQUASILINEAR}) holds a.e. in ${\mathbb D}.$ Moreover, $u\in
W^{1,q}_{\rm loc}(D)$ for some $q>2$ and $u$ is locally H\"older
continuous in $D$. If in addition $\Phi$ is H\"older continuous,
then $u$ is H\"older continuous in $\overline{D}$. Furthermore, if
$p>2$, then $u\in C^{1,\alpha}_{\rm loc}(D)$, where $\alpha =
(p-2)/p$. In particular, $u\in C^{1,\alpha}_{\rm loc}(D)$ for all
$\alpha\in(0,1)$ if $h\in L^{\infty}(D)$. In the latter case, if in
addition $\Phi$ is H\"older continuous on $\partial D$ with some
order $\beta\in(0,1)$, then $u$ is H\"older continuous in
$\overline{ D}$ with the same order.}

\begin{proof} Let $\omega$ be a conformal mapping of $D$ onto
$\mathbb D$. By the Caratheodory--Osgood--Taylor theorem, $\omega$
is extended to a homeomorphism $\tilde{\omega}$ of $\overline{D}$
onto $\overline{\Bbb D}$, see \cite{C} and \cite{OT}, see also
\cite{Arsove} and Theorem 3.3.2 in the monograph \cite{CL}. Then,
setting $\varphi =\Phi\circ\tilde{\omega}^{-1}|_{\partial\mathbb
D}$, we see that the function $\varphi:\partial\mathbb D\to\mathbb
R$ is continuous. Let $h=J\cdot H\circ\Omega$, where $\Omega$ is the
inverse mapping $\omega^{-1}:\mathbb D\to D$ and $J$ is its Jacobian
$J=|\Omega^{\prime}|^2$. By the known Warschawski result, see
Theorem 2 in \cite{W}, its derivative $\Omega^{\prime}$ is extended
by continuity onto $\overline{\Bbb D}$. Consequently, $J$ is bounded
and the function $h$ is of the same class in $\mathbb D$ as $H$ in
$D$. Let $U$ be a solution of the Dirichlet problem from Proposition
1 for the equation (\ref{eqQUASILINEAR}) with the given $\varphi$
and $h$. Note that $\omega^{\prime}=1/\Omega^{\prime}\circ \omega$
is also extended by continuity onto $\overline D$ because
$\Omega^{\prime}\neq 0$ on $\partial\mathbb D$ by Theorem 1 in
\cite{W}. Thus, $u=U\circ\omega$ is the desired solution of the
Dirichlet problem for the equation (\ref{eqQUASILINEARC}).
\end{proof}

By our factorization theorem, mentioned in the Introduction, the
degree of the regularity of the weak solutions of the semi-linear
equation (\ref{eqQUASI}) will depend on the degree of integrability
of the Jacobian $J$ of the quasiconformal mapping $\Omega : \mathbb
D\to D$ associated with the matrix function $A$, see equation
(\ref{QUASII}). In turn, the latter depends on geometry of the
domain $D$.

%\medskip

{\bf Remark 1.} By Theorem 4.7 in \cite{AK}, the Jacobian of a
qua\-si\-con\-for\-mal mapping $\Omega: \mathbb D\to D$ is in
$L^p(\mathbb D)$, $p>1$, if and only if the domain $D$ satisfies the
{\bf quasihyperbolic boundary condition}, i.e.
\begin{equation} \label{eqHYPERB}
k_D(z,z_0)\ \le\ a~\ln \frac{d(z_0,\partial D)}{d(z ,\partial D)}\,
+\, b\ \ \ \ \ \ \ \ \ \forall\ z\in D
\end{equation}
for some constants $a$ and $b$ and a fixed point $z_0\in D$, where
$k_D(z,z_0)$ is the {\bf quasi\-hyperbolic distance} between the
points $z$ and $z_0$ in the domain $D$,
\begin{equation} \label{eqHYPERD}
k_D(z,z_0)\ :=\ \inf\limits_{\gamma} \int\limits_{\gamma}
\frac{ds}{d(\zeta,\partial D)}.
\end{equation}
Here $d(\zeta,\partial D)$ denotes the Euclidean distance from a
point $\zeta\in D$ to the boun\-da\-ry of $D$ and the infimum is
taken over all rectifiable curves $\gamma$ joining the points $z$
and $z_0$ in $D$.

%\medskip

In this connection, recall more definitions. The image of the unit
disk $\mathbb D$ under a quasiconformal mapping of $\mathbb C$
onto itself is called a {\bf quasidisk} and its boundary is called
a {\bf quasicircle} or a {\bf quasiconformal curve}. Recall also
that a {\bf Jordan's curve} is a continuous one-to-one image of
the unit circle in $\mathbb C$. As known, such a smooth ($C^1$) or
Lipschitz curve is a quasiconformal curve and, at the same time,
quasiconformal curves can be even locally non--rectifiable as it
follows from the well-known Van Koch snowflake example, see e.g.
the point II.8.10 in \cite{LV:book}. The recent book \cite{GH}
contains a comprehensive discussion and numerous characterizations
of quasidisks, see also \cite{Ahlfors:book, Ge} and
\cite{LV:book}.

%\medskip
 {\bf Remark 2.} Quasidisks satisfy the quasihyperbolic boundary
condition. Indeed, as known, the conformal mapping $\Omega : \mathbb
D\to D$ is extended to a quasiconformal mapping of $\mathbb C$ onto
itself if $\partial D$ is a quasicircle, see e.g. Theorem II.8.3 in
\cite{LV:book}. By one of the main Bojarski results, see \cite{Bo},
the generalized derivatives of quasiconformal mappings in the plane
are locally integrable with some power $q>2$. Note also that its
Jacobian $J(w)=|\Omega_w|^2-|\Omega_{\bar{w}}|^2$, see e.g. I.A(9)
in \cite{Ahlfors:book}. Consequently, in this case $J\in L^p(\mathbb
D)$ for some $p>1$.

%\medskip

A domain $D$ in ${\mathbb R}^n$, $n\ge 2$, is called satisfying {\bf
(A)-condition} if
\begin{equation} \label{eqA}
\hbox{mes}\ D\cap B(\zeta,\rho)\ \le\ \Theta_0~\hbox{mes}\
B(\zeta,\rho)\ \ \ \ \ \ \ \ \forall\ \zeta\in\partial D\ ,\
\rho\le\rho_0
\end{equation}
for some $\Theta_0$ and $\rho_0\in(0,1)$, see 1.1.3 in \cite{LU}.
Recall also that a domain $D$ in ${\mathbb R}^n$, $n\ge 2$, is said
to be satisfying the {\bf outer cone condition} if there is a cone
that makes possible to be touched by its top to every boundary point
of $D$ from the completion of $D$ after its suitable rotations and
shifts. It is clear that the outer cone condition implies
(A)-condition.

%\medskip

 {\bf Remark 3.} Note that quasidisks $D$ satisfy (A)-condition.
Indeed, the quasidisks are the so-called $QED$-domains by
Gehring--Martio, see Theorem 2.22 in \cite{GM}, and the latter
satisfy the condition
\begin{equation} \label{eqQ} \hbox{mes}\ D\cap B(\zeta,\rho)\ \ge\
\Theta_*~\hbox{mes}\ B(\zeta,\rho)\ \ \ \ \ \ \ \ \forall\
\zeta\in\partial D\ ,\ \rho\le\hbox{diam} D
\end{equation}
for some $\Theta_*\in(0,1)$, see Lemma 2.13 in \cite{GM}, and
quasidisks (as domains with quasihyperbolic boundary) have
boundaries of the Lebesgue measure zero, see e.g. Theorem 2.4 in
\cite{AK}. Thus, it remains to note that, by definition, the
completions of quasidisks $D$ in the the extended complex plane
$\overline{\mathbb C}:=\mathbb C\cup\{\infty\}$ are also
quasidisks up to the inversion with respect to a circle in $D$.

%\medskip

Probably the first example of a simply connected plane domain $D$
with the quasihyperbolic boundary condition which is not a quasidisk
was constructed in \cite{BP}, Theorem 2. However, this domain
satisfies  (A)-condition. In the next section, we construct an
example of a domain $D$ with the quasihyperbolic boundary condition
but without (A)-condition and, consequently, without the outer cone
condition, see Lemma 1.




\section{Dirichlet problem for semi--linear equations}

By the mentioned above  factorization theorem from \cite{GNR2017},
the study of semi--linear equations (\ref{eqQUASI}) {\it in Jordan
domains} $D$ is reduced, by means of a suitable quasiconformal
change of variables, to the study of the corresponding quasilinear
Poisson equations (\ref{QUASII}) {\it in the unit disk} $\mathbb
D$.



\begin{theorem}
Let $D$ be a Jordan's domain in $\mathbb C$
satisfying the quasihyperbolic boun\-da\-ry condition. Suppose
that $A\in M^{2\times 2}_K(D)$, $\varphi :\partial D\to\mathbb R$
is a con\-ti\-nu\-ous function and $f:\mathbb R\to\mathbb R$ is a
continuous function which is either bounded or with nondecreasing
$|f\,|$ of $\ |t|$ such that
\begin{equation} \label{eqEQ}
\lim\limits_{t\to +\infty}\ \frac{f(t)}{t}\ =\ 0.
\end{equation}
Then there is a weak solution $u:{D}\to\mathbb
R$ of the equation (\ref{eqQUASI}) which is locally H\"older
continuous in $D$ and continuous in $\overline{D}$ with
$u|_{\partial D}=\varphi$. If in addition $\varphi$ is H\"older
continuous, then $u$ is H\"older continuous in $\overline{ D}$.
\end{theorem}

\begin{proof}  By Theorem 4.1 in \cite{GNR2017}, if $u$ is a week
solution of (\ref{eqQUASI}), then $u=U\circ\omega$, where $\omega $
is a quasiconformal mapping of $D$ onto the unit disk $\mathbb D$
agreed with $A$ and $U$ is a week solution of the equation
(\ref{eqQUASILINEAR}) with $h=J$, where $J$ stands for the Jacobian
of $\omega^{-1}$. It is also easy to see that if $U$ is a week
solution of (\ref{eqQUASILINEAR}) with $h=J$, then $u=U\circ\omega$
is a week solution of (\ref{eqQUASI}). It allows us to reduce the
Dirichlet problem for equation (\ref{eqQUASI}) with a continuous
boundary function $\varphi$ in the simply connected Jordan domain
$D$  to the Dirichlet problem for the equation (\ref{eqQUASILINEAR})
in the unit disk $\mathbb D$ with the continuous boundary function
$\psi=\varphi\circ\omega^{-1}$. Indeed, $\omega$ is extended to a
ho\-meo\-mor\-phism of $\overline D$ onto $\overline{\Bbb D}$, see
e.g. Theorem I.8.2 in \cite{LV:book}. Thus, the function $\psi$ is
well defined and really is continuous on the unit circle.

It is well-known that the quasiconformal mapping $\omega$ is
locally H\"older continuous in $D$, see  Theorem 3.5 in
\cite{Bo^*}. Taking into account the fact that $D$ is a Jordan's
domain in $\mathbb C$ satisfying the quasihyperbolic boun\-da\-ry
condition, we can show that both mappings $\omega$ and
$\omega^{-1}$ are H\"older continuous in $\overline{D}$ and
$\overline{\mathbb D}$, correspondingly. Indeed,
 $\omega =
H\circ \Omega$ where $\Omega$ is a conformal (Riemann) mapping of
$D$ onto $\mathbb D$ and $H$ is a quasiconformal mapping of $\mathbb
D$ onto itself. The mappings $\Omega$ and $\Omega^{-1}$ are H\"older
continuous in $\overline{D}$ and in $\overline{\mathbb D}$,
correspondingly, by Theorem 1 and its corollary in \cite{BP}. Next,
by the reflection principle $H$ can be extended to a quasiconformal
mapping of $\mathbb C$ onto itself, see e.g. I.8.4 in
\cite{LV:book}, and, consequently, $H$ and $H^{-1}$ are also
H\"older continuous in $\overline{\mathbb D}$, see again Theorem 3.5
in \cite{Bo^*}. The H\"older continuity of $\omega$ and
$\omega^{-1}$ in closed domains follows immediately.

 Now it is easy
to see that if $\varphi$ is H\"older continuous, then $\psi$ is also
so, and all the conclusions of Theorem 1 follow from Proposition 1.
\end{proof}


\begin{lemma}
There exists a Jordan's domain $D$ in $\mathbb
C$ with the quasihyperbolic boundary condition, that does not
satisfy either condition (A) or the condition of the outer cone.
\end{lemma}

\begin{proof} Let $C_1$ be the cube $\{z=x+iy:\ |x|<1,\ |y|<1\
\}$, $R_n$ are the rectangles $\{ z=x+iy:\ 1\le x<a_n,\
\varepsilon_n \le y<\varepsilon_{n-1}\ \}$ with $a_n=1/n$ and
$\varepsilon_n=2^{-n}$ and $R_n^*$ are the reflections of $R_n$ with
respect to the real axis, $n=1,2,\ldots$. Let $D$ be the domain
consisting of the cube $C_1$ and the remainder
$R:=\bigcup\limits_{n=1}^{\infty}(R_n\cup R_n^*)$. First of all, it
is clear that $D$ is a Jordan's domain whose boundary consists of a
countable collection of segments of horizontal and vertical straight
lines and the point $z_0=1$.

Let us show that $D$ satisfies the quasihyperbolic boundary
condition. Note firstly that the quasihyperbolic distance from $0$
to any point in its central closed cube $C_{3/4}:=\{z=x+iy:\
|x|\le\frac{3}{4},\ |y|\le 3/4 \}$ is not grater than $3$. Now, let
$C$ be the continuum consisting of the cube $C_{3/4}$ and the
segments $3/4\le x\le 7/4$ on the straight lines $y=3/4$ and
$y=-3/4$. It is clear by the triangle inequality that the
quasihyperbolic distance from $0$ to any point of $C$ is not grater
than $7$.

Next, note that all points in the triangle $\triangle$ with the
vertices $7/4+ 3i/4$, $2+i$ and $2+i/2$ lie more closely to the
vertical line $x=2$ than to the horizontal lines $y=1$ and $y=1/2$
because its sides $\left(7/4+ 3i/4, 2+i\right)$ and $\left(7/4+
3i/4, 2+i/2\right)$ are bisectrices of the right angles at vertices
$2+i$ and $2+i/2$ of $\partial D$. Each point in $\triangle$ lies on
a segment of a straight line starting from the corresponding point
on the side $\left(2+i,2+i/2\right)\subset\partial D$ and ending at
the point $7/4+ 3i/4\in C$ and the slope of the line to the side is
varied in the limits $\pi/4$ and $\pi/2$. Let $s$ be the natural
parameter on one of such segments $S$ with $s=0$ at the
corresponding point of $\partial D$ and $\zeta(s)$ be the natural
parametrization of points on $\gamma$. Then
\begin{equation} \label{eqLOWER}
d(\, \zeta(s)\, ,\, \partial D\, )\ \ge\ s/{\surd 2}.
\end{equation}
By the symmetry of $D$, the similar statement is true for the
triangle $\triangle^*$ that is symmetric for $\triangle$ with
respect to the real axes.

Note also that every point in $D\setminus C$, except the points of
the triangles $\triangle$ and  $\triangle^*$, lies on a segment of a
straight line going under the angle $\pi/4$ with respect to
horizontal and vertical straight lines, starting from the
corresponding point on $\partial D$ and ending at the nearest point
on the continuum $C$. It is clear that (\ref{eqLOWER}) holds on such
segments, too. The lengths of all segments mentioned above are
bounded by the diameter $\delta $ of $D$, $\delta =\surd 13\le 4$
and, consequently, $k_D(\zeta(s_0),\zeta(s_*))\le \surd 2~(\ln s_* -
\ln s_0)\le \surd 2~(\ln \delta - \ln s_0)$, where $s_*$ and $s_0$
correspond to points in $C$ and in $D\setminus C$. Thus, by the
triangle inequality
$$
k_D(z,0) \le \surd 2~ \ln\frac{d(0,\partial D)}{d(z,\partial D)} + 7
+ \surd 2~ \ln\frac{\delta}{\surd 2} < \surd 2\ln\frac{d(0,\partial
D)}{d(z,\partial D)} + 10\ \ \ \forall z\in D,
$$
i.e. the domain $D$ really satisfies the quasihyperbolic boundary
condition.

Finally, let us show that $D$ does not satisfy $A$--condition at
the point $(1,0)$. Indeed, let us consider the sequence of disks
$D_n$ centered at the given point with the radii
$\rho^2_n=a_n^2+\varepsilon_n^2={n^{-2}}+2^{-2n}$. Note that
$D_n\cap D$ contains 2 caps of the disk $D_n$ that are cut off by
the horizontal straight lines $y=\varepsilon_n=2^{-n}$ and
$y=-\varepsilon_n=-2^{-n}$. Consequently,
$$
{\rm I}_n\ :=\ \frac{\hbox{mes}\, (D_n\cap D)}{\hbox{mes}\, D_n}\
\ge\ \frac{\alpha_n\ -\ \sin \alpha_n}{\pi},
$$
where $\alpha_n$ is the angular size of each of these caps. Since
$\sin(\alpha_n/2)={a_n}/{\rho_n}$ converges to $1$ as $n\to\infty$,
we have that $\alpha_n\to\pi$, i.e. I$_n$ converges to $1$ as
$n\to\infty.$
\end{proof}

%\medskip

Scaling, rotating and shifting the remainder $R$ from the proof, it
is possible to construct Jordan domains $D$ that are similar to the
Van Koch snowflake with the quasihyperbolic boundary condition and,
simultaneously, without (A)-condition at the everywhere dense set of
boundary points.



\section{The Dirichlet problem in terms of prime ends}

It is much more simpler than in Lemma 1 to construct similar
examples of domains with  the quasihyperbolic boundary condition
that are not Jordan.

%\medskip

\begin{lemma}  There exist bounded simply connected domains
$D_*$ in $\mathbb C$ that are not Jordan, satisfy the
quasihyperbolic boundary condition, however, without (A)-condition
and, consequently, without the outer cone condition.
\end{lemma}

%\medskip

\begin{proof} Let $P$ be the rectangular $\{z=x+iy:\ -1<x<2,\
|y|<1\}$ and let the domain $D$ be obtained from $P$ through cut
along $1\le x< 2$ in the real axis. Denote by $C$ the union of the
cube $C_{1/2}:=\{z=x+iy:\ |x|\le 1/2,\ |y|\le 1/2 \}$ and 2 segments
$1/2\le x\le 3/2$ on the straight lines $y=1/2$ and $y=-1/2$. Let
$\triangle$ be the triangle with the vertices $3/2+ i/2$, $2+i$ and
$2$ and let $\triangle^*$ be the triangle which is symmetric for
$\triangle$ with respect to the real axis. Arguing as in the proof
of Lemma 2, it is easy to get the following estimate $$ k_D(z,0)\ <\
\surd 2 \ln\frac{d(0,\partial D)}{d(z,\partial D)}\ +\ 5\ \ \ \ \ \
\ \forall\ z\in D\,
$$
i.e. the domain $D$ is really with the quasihyperbolic boundary
condition, but it is clear that (A)-condition does not hold at the
end point of the cut in $P$.
\end{proof}

Before to formulate the corresponding results for non-Jordan
domains, let us recall the necessary de\-fi\-ni\-tions of the
relevant notions and notations. Namely, we follow Caratheodory
\cite{Car$_2$} under the definition of the prime ends of domains in
$\Bbb C$, see also Chapter 9 in \cite{CL}. First of all, recall that
a continuous mapping $\sigma: \Bbb I\to \Bbb C$, $\Bbb I=(0,1)$, is
called a {\bf Jordan arc} in $\Bbb C$ if
$\sigma(t_1)\neq\sigma(t_2)$ for $t_1\neq t_2$. We also use the
notations $\sigma$, $\overline{\sigma}$ and $\partial\sigma$ for
$\sigma(\Bbb I)$, $\overline{\sigma(\Bbb I)}$ and
$\overline{\sigma(\Bbb I)}\setminus\sigma(\Bbb I)$, correspondingly.
A {\bf cross--cut} of a simply connected domain $D\subset\Bbb C$ is
a Jordan arc $\sigma$ in the domain $D$ with both ends on $\partial
D$ splitting $D$.

A sequence $\sigma_1,\ldots, \sigma_m,\ldots$ of cross-cuts of $D$
is called a {\bf chain} in $D$ if:

(i) $\overline{\sigma_i}\cap\overline{\sigma_j}=\varnothing$ for
every $i\neq j$, $i,j= 1,2,\ldots$;

(ii) $\sigma_{m}$ splits $D$ into 2 domains one of which contains
$\sigma_{m+1}$ and another one $\sigma_{m-1}$ for every $m>1$;

(iii)  $\delta(\sigma_{m})\to0$ as $m\to\infty$ where
$\delta(\sigma_{m})$ is the diameter of $\sigma_m$ with respect to
the Euclidean metric in $\mathbb C$.

Correspondingly to the definition, a chain of cross-cuts $\sigma_m$
generates a sequence of do\-mains $d_m\subset D$ such that
$d_1\supset d_2\supset\ldots\supset d_m\supset\ldots$ and $\,
D\cap\partial\, d_m=\sigma_m$. Chains of cross-cuts $\{\sigma_m\}$
and $\{\sigma_k'\}$ are called {\bf equivalent} if, for every
$m=1,2,\ldots$, the domain $d_m$ contains all domains $d_k'$ except
a finite number and, for every $k=1,2,\ldots$, the domain $d_k'$
contains all domains $d_m$ except a finite number, too. A {\bf prime
end} $P$ of the domain $D$ is an equivalence class of chains of
cross-cuts of $D$. Later on, $E_D$ denote the collection of all
prime ends of a domain $D$ and $\overline D_P=D\cup E_D$ is its
completion by its prime ends.

Next, we say that a sequence of points $p_l\in D$ is {\bf convergent
to a prime end} $P$ of $D$ if, for a chain of cross-cuts $\{
\sigma_m\}$ in $P$, for every $m=1,2,\ldots$, the domain $d_m$
contains all points $p_l$ except their finite collection. Further,
we say that a sequence of prime ends $P_l$ converge to a prime end
$P$ if, for a chain of cross--cuts $\{ \sigma_m\}$ in $P$, for every
$m=1,2,\ldots$, the domain $d_m$ contains chains of cross--cuts $\{
\sigma_k'\}$ in all prime ends $P_l$ except their finite collection.

A basis of neighborhoods of a prime end $P$ of $D$ can be defined
in the following way. Let $d$ be an arbitrary domain from a chain
in $P$. Denote by $d^*$ the union of $d$ and all prime ends of $D$
having some chains in $d$. Just all such $d^*$ form a basis of
open neighborhoods of the prime end $P$. The corresponding
topology on $E_D$ and, respectively, on $\overline D_P$ is called
the {\bf topology of prime ends}. The continuity of functions on
$E_D$ and $\overline D_P$ will be understood with respect to this
topology or, the same, with respect to the above convergence.

\begin{theorem}  Let $D$ be a bounded simply connected domain
in $\mathbb C$ satisfying the quasihyperbolic boun\-da\-ry
condition. Suppose $A\in M^{2\times 2}_K(D)$, $\varphi
:E_D\to\mathbb R$ is a con\-ti\-nu\-ous function and $f:\mathbb
R\to\mathbb R$ is a continuous function which is either bounded or
with nondecreasing $|f\,|$ of $\ |t|$ such that
\begin{equation} \label{eqEQE}
\lim\limits_{t\to +\infty}\ \frac{f(t)}{t}\ =\ 0.
\end{equation}
Then there is a weak solution $u:{D}\to\mathbb
R$ of the equation (\ref{eqQUASI}) which is locally H\"older
continuous in $D$ and continuous in $\overline{D}_P$ with
$u|_{E_D}=\varphi$.
\end{theorem}


\begin{proof}  By Theorem 4.1 in \cite{GNR2017}, if $u$ is a week
solution of (\ref{eqQUASI}), then $u=U\circ\omega,$ where $\omega
$ is a quasiconformal map of $D$ onto the unit disk $\mathbb D$
agreed with $A$ and $U$ is a week solution of the equation
(\ref{eqQUASILINEAR}) with $h=J$, here $J$ stands for the Jacobian
of the mapping $\omega^{-1}.$ And vice versa,  if $U$ is a week
solution of (\ref{eqQUASILINEAR}) with $h=J$, then
$u=U\circ\omega$ is a week solution of (\ref{eqQUASI}).

Hence the Dirichlet problem for (\ref{eqQUASI}) in the domain $D$
can be reduced to the so for the equation (\ref{eqQUASILINEAR}) in
$\mathbb D$ with the corresponding boundary function
$\psi=\varphi\circ\omega^{-1}$. The existence and continuity of the
boundary function $\psi$ in the case of an arbitrary bounded simply
connected domain $D$ is a fundamental result of the theory of the
boundary behavior of conformal and quasiconformal mappings. Namely,
$\omega^{-1}=H\circ \Omega$, where $\Omega$ stands for a
quasiconformal automorphism of the unit disk ${\Bbb D}$ and $H$ is a
conformal mapping of $\Bbb D$ onto $\Omega.$ It is known that
$\Omega$ can be extended to a homeomorphism of $\overline{\Bbb D}$
onto itself, see e.g. Theorem I.8.2 in \cite{LV:book}. Moreover, by
the well-known Caratheodory theorem on the boundary correspondence
under conformal mappings, see e.g. Theorems 9.4 and 9.6 in
\cite{CL}, the mapping $H$ is extended to a homeomorphism of
$\overline{\Bbb D}$ onto $\overline{D}_P$. Thus, the function $\psi$
is well defined and really continuous on the unit circle.

Moreover, $\omega$ is locally H\"older continuous in $\mathbb D$,
see e.g. Theorem 3.5 in \cite{Bo^*}. Thus,  by Theorem 4.1 in
\cite{GNR2017}, Theorem 2 follows from Proposition 1.
\end{proof}




\section{Some model equations}

The interest to the study of some model semi-linear equations
considered below  is well known both from a purely theoretical
point of view, due to its deep relations to linear and nonlinear
harmonic analysis, and because of numerous applications of
equations of this type  in various areas of physics, differential
geometry, logistic problems etc., see e.g. \cite{AIM, BK, Diaz, GT, KP, Landis, MV, NN} and the references therein.

In particular, the mathematical modelling of reaction-diffusion
problems leads to the study of the corresponding Dirichlet problem
for the equation (\ref{eqQUASI}) with specified right hand side.
Following \cite{Aris},  a nonlinear system can be obtained for the
density $u$ and the temperature $T$ of the reactant. Upon
eliminating $T$ the system can be reduced to a scalar problem for
the concentration
\begin{equation}\label{RDP}
 \triangle  u\ =\ \lambda~f(u),
 \end{equation}
where $\lambda$ stands for a positive constant. It turns out that
the density of the reactant $u$ may be zero in a closed interior
region $D_0$ called {\it a dead core.} If, for instance, $f(u) =
u^q,$  $q > 0,$ a particularization of the results in Chapter 1 of
\cite{Diaz} shows that a dead core may only exist if and only if
$0 < q < 1$ and $\lambda$ is large enough. See also the
corresponding examples of  cores in \cite{GNR2017}. In connection
with the above, the following statement may have of independent
interest.


\begin{theorem}  Let $D$ be a Jordan's domain in $\mathbb C$
satisfying the quasihyperbolic boundary condition. Suppose that
$A\in M^{2\times 2}_K(D)$ and $\varphi :\partial D\to\mathbb R$ is
a con\-ti\-nu\-ous function. Then there exists a weak solution
$u:{D}\to\mathbb R$ of the semi-linear equation
\begin{equation} \label{eqQUASIQ}
{\rm div\,}[\, A(z)\, \nabla u(z)\, ]\ =\ u^q(z), \ \ \ 0\ <\ q\ <\
1,
\end{equation}
which is locally H\"older continuous in $D,$  continuous in
$\overline{D}$  and satisfies the boundary condition $u|_{\partial
D}=\varphi.$ If in addition $\varphi$ is H\"older continuous, then
$u$ is H\"older continuous in $\overline{D}$.
\end{theorem}


Applying Corollary 1, we also arrive at the following  consequence.


 {\bf Corollary 2.} {\it Let $D$ be a smooth Jordan's domain in
$\mathbb C$ and let $\varphi :\partial D\to\mathbb R$ be a
con\-ti\-nu\-ous function. Then there exists a weak solution $U$ of
the quasilinear Poisson equation
\begin{equation}\label{eqQUASILINEARQ}
\triangle\, U(z)\ =\ U^q(z), \ \ \ 0\ <\ q\ <\ 1,
\end{equation}
which is continuous in $\overline{D}$ with $U|_{\partial
D}=\varphi$ and such that $U\in C^{1,\alpha}_{\rm loc}(D)$ for all
$\alpha\in(0,1)$. If in addition $\varphi$ is H\"older continuous
with some order $\beta\in(0,1)$, then $U$ is also H\"older
continuous in $\overline{D}$ with the same order.}

%\medskip

Recall also that certain mathematical models of a heated plasma lead
to nonlinear equations of the type (\ref{RDP}). Indeed, it is known
that some of them have the form $\triangle\psi(u)=f(u)$ with
$\psi'(0)=+\infty$ and $\psi'(u)>0$ if $u\not=0$ as, for instance,
$\psi(u)=|u|^{q-1}u$ under $0 < q < 1$, see e.g. \cite{Diaz}. With
the replacement of the function $U=\psi(u)=|u|^q~ {\rm sign}\, u$,
we have that $u = |U|^Q~ {\rm sign}\, U$, $Q=1/q$, and, with the
choice $f(u) = |u|^{q^2}~ {\rm sign}\, u$, we come to the equation
$\triangle U = |U|^q~{\rm sign}\, U=\psi(U)$.


{\bf Corollary 3.} {\it Let $D$ be a smooth Jordan's domain in
$\mathbb C$ and let $\varphi :\partial D\to\mathbb R$ be a
con\-ti\-nu\-ous function. Then there exists a weak solution $U$ of
the quasilinear Poisson equation
\begin{equation}\label{eqPLASMA}
\triangle\, U(z)\ =\ |U(z)|^{q-1}U(z), \ \ \ 0\ <\ q\ <\ 1,
\end{equation}
which is continuous in $\overline{D}$ with $U|_{\partial
D}=\varphi$ and such that $U\in C^{1,\alpha}_{\rm loc}(D)$ for all
$\alpha\in(0,1)$. If in addition $\varphi$ is H\"older continuous
with some order $\beta\in(0,1)$, then $U$ is also H\"older
continuous in $\overline{ D}$ with the same order. }

%\medskip

In the combustion theory, see e.g. \cite{Barenblat},
\cite{Pokhozhaev} and the references therein, the following model
equation
\begin{equation}
{\partial u(z,t)\over \partial t}\ =\ {1\over \delta}~ \triangle u\
+\ e^{u},\ \ \ t\geq 0,\ z\in D,
\end{equation}
occupies a special place. Here $u\ge 0$ is the temperature of the
medium and $\delta$ is a certain positive parameter.

We restrict ourselves by stationary solutions of the equation and
its ge\-ne\-ra\-li\-za\-tions in anisotropic and inhomogeneous
media although our approach makes it possible to consider the
parabolic case, see \cite{GNR2017}. Applying  Theorem 1, we come
to the following statement.

%\medskip

\begin{theorem}  Let $D$ be a Jordan's domain in $\mathbb C$
satisfying the quasihyperbolic boun\-da\-ry condition. Suppose
that $A\in M^{2\times 2}_K(D)$ and $\varphi :\partial D\to\mathbb
R$ is a con\-ti\-nu\-ous function. Then there exists a weak
solution $U:{D}\to\mathbb R$ of the semi-linear equation
\begin{equation} \label{eqQUASIQA}
{\rm div\,}[\, A(z)\, \nabla U(z)\, ]\ =\ \delta~e^{-U(z)}\ , \ \ \
\delta > 0,
\end{equation}
which is locally H\"older continuous in $D,$  continuous in
$\overline{D}$ and such that $u|_{\partial D}=\varphi$. If in
addition $\varphi$ is H\"older continuous, then $u$ is also
H\"older continuous in $\overline{ D}$.
\end{theorem}

By Corollary 1, applied to the corresponding quasilinear Poisson
equation, we  will finish this section with the following
statement.

%\medskip







{\bf Corollary 4.} {\it Let $D$ be a smooth Jordan's domain in
$\mathbb C$ and $\varphi :\partial D\to\mathbb R$ be a
con\-ti\-nu\-ous function. Then there is a weak solution $U$ of the
equation
\begin{equation}\label{eqQUASILINEARQ}
\triangle\, U(z)\ =\ \delta~ e^{-U(z)}\ , \ \ \ \delta > 0,
\end{equation}
which is continuous in $\overline{D}$ with $U|_{\partial
D}=\varphi$ and such that $U\in C^{1,\alpha}_{\rm loc}(D)$ for all
$\alpha\in(0,1)$. If in addition $\varphi$ is H\"older continuous
with some order $\beta\in(0,1)$, then $U$ is also H\"older
continuous in $\overline{D}$ with the same order.}

%\medskip

Concluding the presentation, we want to emphasize the fact that
joint use of the regularity results for the quasilinear Poisson
equations (\ref{eqQUASILINEAR}) and the comprehensively developed
theo\-ry of conformal and quasiconformal mappings in the plane,
see e.g. the monographs \cite{Ahlfors:book, BGMR, GR2011, GRSY, LV:book, MRSY, Vuorinen} opens up a new approach to the study of a number
of problems arising in the mathematical physics in anisotropic and
inhomogeneous media.

\begin{thebibliography}{99}
\small

\bibitem{Ahlfors:book}
\refitem{book}
L.~V. Ahlfors, {\em Lectures on quasiconformal mappings}, Van
Nostrand Ma\-the\-ma\-ti\-cal Studies, 10  (Van Nostrand Co.,
Inc., Toronto, Ont.-New York-London, 1966).

\bibitem{AIM}
\refitem{book}
K. Astala, T. Iwaniec and G.~J. Martin, {\em Elliptic Partial
Di�erential Equations and Quasiconformal Mappings in the Plane},
Princeton Math. Ser., 48  (Princeton Univ. Press, Princeton,
2009).


\bibitem{AK}
\refitem{article} K. Astala and P. Koskela, \textquotedblleft
Quasiconformal mappings and global integrability of the
derivative,\textquotedblright  J. Anal. Math. {\bf 57}, 203--220
(1991).

\bibitem{Aris}
\refitem{book} R. Aris, {\em The Mathematical Theory of Diffusion
and Reaction in Permeable Ca\-ta\-lysts}, Volumes I and II
(Clarendon Press, Oxford, 1975).

\bibitem{Arsove}
\refitem{article}
M.~G. Arsove, \textquotedblleft The Osgood-Taylor-Caratheodory theorem,\textquotedblright  Proc.
Amer. Math. Soc. {\bf 19}, 38--44 (1968).

\bibitem{Barenblat}
\refitem{book} G.~I. Barenblatt, Ja.~B. Zel'dovic, V.~B. Librovich,
and G.~M. Mahviladze, \emph{The mathematical theory of combustion
and explosions} (Consult. Bureau, New York, 1985).

\bibitem{BP}
\refitem{article} J. Becker, Ch. Pommerenke, \textquotedblleft
H\"older continuity of conformal mappings and nonquasiconformal
Jordan curves,\textquotedblright  Comment. Math. Helv. {\bf 57} (2),
221--225 (1982).


\bibitem{Bo}
\refitem{article}
B.~V. Bojarski, \textquotedblleft Homeomorphic solutions of Beltrami systems,\textquotedblright
 Dokl. Akad. Nauk SSSR (N.S.) {\bf 102}, 661--664 (1955)  [in Russian].

\bibitem{Bo^*}
\refitem{article} B.~V. Bojarski, \textquotedblleft Generalized
solutions of a system of differential equations of the first order
and elliptic type with discontinuous coefficients,\textquotedblright
Mat. Sb. (N.S.) {\bf 43} (85), 451--503  (1957) [in Russian].

\bibitem{BGMR}
\refitem{book} B. Bojarski, V. Gutlyanskii, O. Martio, and V.
Ryazanov, {\em Infinitesimal geometry of quasiconformal and
bi-lipschitz mappings in the plane}, EMS Tracts in Ma\-the\-ma\-tics
{19} (European Mathematical Society, Z\"urich, 2013).

\bibitem{BK}
\refitem{book} M. Borsuk and V. Kondratiev,  {\em Elliptic boundary
value problems of second order in piecewise smooth domains},
North-Holland Mathematical Library {69} (Elsevier Science,
Amsterdam, 2006).

\bibitem{C}
\refitem{article}
C. Caratheodory, \textquotedblleft Uber die gegenseitige Beziehung der Rander
bei der konformen Abbildungen des Inneren einer Jordanschen Kurve
auf einen Kreis,\textquotedblright Math. Ann. {\bf 73}, 305--320 (1913).

\bibitem{Car$_2$}
\refitem{article}
C. Caratheodory, \textquotedblleft  \"{U}ber die Begrenzung der
einfachzusammenh\"{a}ngender Gebiete,\textquotedblright   Math. Ann. \textbf{73},  323--370 (1913).

\bibitem{CL}
\refitem{book} E.~F. Collingwood and A.~J. Lohwator, {\em The Theory
of Cluster Sets}, Cambridge Tracts in Math. and Math. Physics, {56}
(Cambridge Univ. Press, Cambridge, 1966).

\bibitem{Diaz}
\refitem{book} J.~I. Diaz, {\em Nonlinear partial differential
equations and free boundaries}, Volume I Elliptic equations,
Research Notes in Mathematics {106} (Pitman, Boston, 1985).

\bibitem{Ge}
F.~W. Gehring, \emph{Characteristic properties of quasidisks},
Seminaire de Math�ematiques Superieures {84} (Presses de
l�Universit�e de Montr�eal, Montreal, Que., 1982).

\bibitem{GH}
\refitem{book} F.~W. Gehring and K. Hag, {\em The ubiquitous
quasidisk}, Mathematical Surveys and Monographs {184} (American
Mathematical Society, Providence, RI, 2012).

\bibitem{GM}
\refitem{article}
F.~W. Gehring and O. Martio, \textquotedblleft Quasiextremal distance domains and
extension of quasiconformal mappings,\textquotedblright   J. Analyse Math. {\bf 45}, 181--206 (1985).

\bibitem{GT}
\refitem{book} D. Gilbarg and N. Trudinger, {\em Elliptic partial
differential equations of second order}, Grundlehren der
Mathematischen Wissenschaften {224} (Springer-Verlag, Berlin, 1983).


\bibitem{GNR2017}
\refitem{article} V.~Ya. Gutlyanskii, O.~V. Nesmelova, and V.~I.
Ryazanov, \textquotedblleft  On quasiconformal maps and semi-linear
equations in the plane,\textquotedblright  Ukr. Mat. Visn. {\bf 14}
(2), 161--191 (2017); transl. in J. Math. Sci. (US) {\bf 229} (1),
7--29 (2018).

\bibitem{GNR2018}
\refitem{article} V.~Ya. Gutlyanskii, O.~V. Nesmelova, and V.~I.
Ryazanov, \textquotedblleft  On the Dirichlet problem for
quasilinear Poisson equations,\textquotedblright  Proceedings of
Inst. Appl. Math. Mech., NAS of Ukraine {\bf 31}, 28--37 (2017).

\bibitem{GR2011}
\refitem{book} V.~Ya. Gutlyanskii and V.~I. Ryazanov, {\em The
geometric and topological theory of functions and mappings} (Naukova
Dumka, Kiev, 2011) [in Russian].


\bibitem{GRSY}
\refitem{book} V. Gutlyanskii, V. Ryazanov, U. Srebro, and E.
Yakubov, {\em The Beltrami Equation: A Geometric Approach},
Developments in Mathematics, {26}  (Springer, New York etc., 2012).

\bibitem{KP}
\refitem{book}
I. Kuzin and S. Pohozaev, {\em Entire solutions of semi-linear
elliptic equations}, Progress in Nonlinear Differential Equations
and their Applications,  33  (Birkh\"auser Verlag, Basel,
1997).

\bibitem{Landis}
\refitem{book}
E.~M. Landis, {\em Second order equations of elliptic and parabolic
type}, Translations of Mathematical Monographs  171
(American Mathematical Society, Providence, RI, 1998).

\bibitem{LU}\refitem{book}
O.~A. Ladyzhenskaya and N.~N. Ural'tseva, {\em Linear and quasilinear
elliptic equations} (Academic Press, New York--London, 1968).

\bibitem{LV:book}
\refitem{book}
O. Lehto and K.~I. Virtanen, {\em Quasiconformal mappings in the
plane,} 2-nd Edition (Springer--Verlag, Berlin--Heidelberg--New
York, 1973).


\bibitem{MV}
\refitem{book}
M. Marcus and L. Veron, {\em Nonlinear second order elliptic
equations involving measures}, De Gruyter Series in Nonlinear
Analysis and Applications, {21}  (De Gruyter, Berlin, 2014).

\bibitem{MRSY}
\refitem{book} O. Martio, V. Ryazanov, U. Srebro, and E. Yakubov,
{\em Moduli in Modern Mapping Theory}  (Springer, New York, 2009).

\bibitem{NN}
\refitem{article} R.~M. Nasyrov and S.~R. Nasyrov, \textquotedblleft
Convergence of S. A. Khristianovich's approximate method for solving
the Dirichlet problem for an elliptic equation,\textquotedblright
Dokl. Akad. Nauk SSSR {\bf 291} (2), 294--298 (1986) [in Russian].

\bibitem{OT}
\refitem{article}
W. Osgood and E. Taylor, \textquotedblleft Conformal transformations on the
boundaries of their regions of definition,\textquotedblright   Trans. Amer. Math.
Soc. {\bf 14}, 277--298 (1913).

\bibitem{Pokhozhaev}
\refitem{article} S.~I. Pokhozhaev, \textquotedblleft  On an
equation of combustion theory,\textquotedblright   Mat. Zametki {\bf
88} (1), 53--62 (2010); transl. in Math. Notes {\bf 88} (1--2),
48--56 (2010).

\bibitem{Vuorinen}
\refitem{book} M. Vuorinen, \emph{Conformal geometry of quasiregular
mappings}, Lecture Note im Math, (1319)  (Springer-Verlag, Berlin,
1988).


\bibitem{W}
\refitem{article}
S.~E. Warschawski, \textquotedblleft  On differentiability at the boundary in
conformal mapping,\textquotedblright   Proc. Amer. Math. Soc. {\bf 12},
614--620 (1961).

\end{thebibliography}
\end{document}
